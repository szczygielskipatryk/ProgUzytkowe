\documentclass[a4paper,12pt]{article}
\usepackage[MeX]{polski}
\usepackage[utf8]{inputenc}
\usepackage{graphicx} 


%opening
\title{Fumio Nutahara}
\author{}

\begin{document}

\maketitle
\section{O osobie}
\label {o_nim}
,,Fumio Nutahara ur. 20 grudnia 1963 w Kochi (prefektura Kochi) – japoński kierowca rajdowy. W swojej karierze siedmiokrotnie zdobył tytuł mistrza Japonii. Był też wicemistrzem świata w serii Production Cars WRC.''
\section{Poczatki}
\label{pocz}
,,W 1999 r. Nutahara zadebiutował w Rajdowych Mistrzostwach Świata. Pilotowany przez Noriyukiego Odagiri i jadący Mitsubishi Lancerem Evo 5 nie ukończył wówczas Rajdzie Nowej Zelandii z powodu awarii.''
\section{Osiągnięcia}
\label{osi}
,,W 2004 r. zaliczył cykl startów w serii Production Cars WRC. Zajął w niej 9. miejsce, a w 2005 r. --- był w niej 4. (zajął 2. miejsce w PCWRC w Rajdzie Japonii). W 2006 r. wywalczył wicemistrzostwo Production Cars. Zdobył 35 punktów, o 5 mniej niż zwycięzca, Katarczyk Nasir al-Atijja. Japończyk wygrał w PCWRC Rajd Monte Carlo, Rajd Japonii i Rajd Cypru.W latach 2007–2008 zajmował 7. pozycję w Production Cars.''
\section {Debiut}
\label{deb}
,,Swój debiut rajdowy Nutahara zaliczył w 1986 r. w wieku 23 lat. Siedmiokrotnie w swojej karierze wywalczył tytuł rajdowego mistrza Japonii. W 2003 r. zajął 2. miejsce w grupie N, w rajdowym Pucharze Azji i Pacyfiku.''
\begin{table}
\begin {tabular}{lccc}
\hline
\textbf{sezon}&\textbf {zespół}&\textbf{Samochód}&\textbf{Punkty}\\
\hline
1999&Advan-PIAA Rally Team&Mitsubishi Lancer Evo 5&0\\
\hline
2000&Advan-PIAA Rally Team&Mitsubishi Lancer Evo 6&0\\
\hline
2001&Advan-PIAA Rally Team&Mitsubishi Lancer Evo 7&0\\
\hline
2004&Advan-PIAA Rally Team&Mitsubishi Lancer Evo 7&0\\
\hline
2005&Advan-PIAA Rally Team&Mitsubishi Lancer Evo 8&0\\
\hline
2006&Advan-PIAA Rally Team&Mitsubishi Lancer Evo 9&1\\
\hline
2007&Advan-PIAA Rally Team&Mitsubishi Lancer Evo 9&0\\
\hline
2008&Advan-PIAA Rally Team&Mitsubishi Lancer Evo 9&0 \\
\hline
\end{tabular}
\caption{Udziały w Mistrzostwach Świata}\label{table}
\end {table}
\begin{figure}
\includegraphics[scale=0.25]{pics/lancer.jpg}
\caption{Lancer IX używany przez Nutahare podczas Rajdu Japoni}\label{fig:lancer}Uzywany także w Mistrzoswtach Świata w 2008 roku 
\end{figure}




\end{document}
