\documentclass{beamer}
\usepackage[MeX]{polski}
\usepackage[utf8]{inputenc}
\usepackage{amsfonts}
\beamersetaveragebackground{blue!10}
\usetheme{Warsaw}

\usepackage{beamerthemesplit}
\usepackage{multirow}
\usepackage{multicol}
\usepackage{array}
\usepackage{graphicx}
\usepackage{enumerate}
\usepackage{amsmath} %pakiet matematyczny
\usepackage{amssymb} %pakiet dodatkowych symboli


%opening
\title{Plewki}
\author{}

\begin{document}

\frame{\titlepage}

\begin {frame}
\frametitle{Spis}
\tableofcontents
\end {frame}


\section{Wstęp}
\begin{frame}{Wstęp}
\begin{itemize}
\item Plewki – niewielka wieś w Polsce położona w województwie podlaskim, w powiecie wysokomazowieckim, w gminie Szepietowo, o charakterze rolniczym.

W latach 1975–1998 miejscowość położona była w województwie łomżyńskim.

\end{itemize}
\end{frame}


\section{Historia}
\begin{frame}{Historia}
\begin {itemize}
\item W roku 1827 we wsi 13 domów i 74 mieszkańców.

W 1886 nomenklatura Plewki częścią folwarku Szepietowo. Pod koniec wieku XIX wieś drobnoszlachecka w powiecie mazowieckim, Gmina Wysokie Mazowieckie, parafia Dąbrowa Wielka


\end {itemize}
\end{frame}
\begin{frame}{Historia c.d}
\begin{itemize}
\item W 1921 r. wieś w Gminie Wysokie Mazowieckie. Naliczono tu 11 budynków z przeznaczeniem mieszkalnym oraz 89 mieszkańców (48 mężczyzn i 41 kobiet). Narodowość polską podało 87 osób, a 2 inną.
\end{itemize}
\end{frame}
\section {Zabytki}
\begin {frame}{Zabytki}
\begin{itemize}
\item Krzyż przydrożny z roku 1904 \cite{deni}
\end{itemize}
\end{frame}
\section {Dojazd}
\begin{frame}{Dojazd}
\begin{itemize}
\item Wieś w odległości ok. 2,5 km na północ od siedziby gminy - Szepietowa i ok. 4 km od miasta powiatowego Wysokie Mazowieckie.

\item Dojazd drogą krajową nr 66. Około 2 km od centrum wsi znajduje się przystanek kolejowy Szepietowo-Stacja.
\end{itemize}
\end{frame}
\begin{frame}
\begin {center}
\begin{figure}
\includegraphics[scale=0.5]{pics/plewki.jpg}
\caption{Położenie Plewek}\label{fig:plewki}
\end{figure}
\end{center}
\end {frame}
\section {Bibliografia}
\begin{frame}{Bibliografia}
\begin{thebibliography}{99}
\bibitem {deni}  Załącznik Nr 1 do uchwały Nr III/13/06 Rady Gminy Szepietowo z dnia 28 grudnia 2006 r.
\end {thebibliography}
\end{frame}


\end{document}